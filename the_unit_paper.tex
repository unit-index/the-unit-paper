\documentclass[12pt]{article}

\usepackage{amsmath,amsfonts,amssymb}   %% AMS mathematics macros

\usepackage{pagecolor}% http://ctan.org/pkg/{pagecolor}

\usepackage{hyperref}

\parindent 0pt
\setlength{\parskip}{1em} 

\title{UNIT: Crypto's Unit of Account v2.0.0}
\author{Ibai Basabe}


\begin{document}


\date{}

\maketitle


\begin{abstract}
We propose an {\bf indexed unit of account} ({\bf UNIT}) built from a decentralized index of the top cryptocurrencies. The birth of an indexed unit would allow the blockchain space to grow from within without needing a fiat currency to quote each blockchain-based currency. With this, we aim to close the gap between blockchain networks and unite the crypto space.  We also propose a UNIT ETF token (TINU) that follows UNIT's value and keeps its peg through overcollateralized vaults. 
\end{abstract}


\tableofcontents
\newpage

\section{Motivation}

\subsection{What is UNIT?}

The current status quo is a series of blockchain networks each with its own currency, in which fiat currencies, mainly USD, are used to quote each base currency. This inherently blocks the transformation envisioned by Satoshi Nakamoto through the invention of the blockchain. When Bitcoin was described in \cite{nakamoto2008bitcoin} as a way to improve the current state of the financial system, Satoshi did not envision the appearance of numerous other blockchain networks that would even increase the original capabilities of Bitcoin. However, 15 years in, we have a rich blockchain ecosystem that has allowed the introduction of on-chain lending markets \cite{compound2019whitepaper} \cite{aave2020whitepaper}, decentralized automated market makers \cite{uniswap2020whitepaper}, and non-fungible tokens \cite{eip721} among many other expansions. Now the blockchain space is engulfing many blockchains with many consensus algorithms and separate validator networks. Now the space is ready for a uniting force, one that is born completely from within.

We desire to establish a crypto-native unit of account that is not significantly affected by any individual cryptocurrency and can reflect the full reach of the crypto market. We achieve this goal by creating a decentralized index of cryptocurrencies. Much like traditional units of account such as the Unidad de Fomento (UF) in Chile \cite{uf2020methodology} and SDR at the IMF \cite{imf2023sdr}, UNIT is not a currency but an indexed unit of account. 

UNIT follows exactly what it would be to hold the top cryptocurrencies by market capitalization, creating a {\bf Hodl Standard}. 

\subsection{Why Should We Care?}

Currently, the cryptocurrency space uses USD to solve the issue of not having a native accounting unit. As a result, coins such as USDT, USDC, DAI and others have become the default unit-of-account utility tokens. On the other hand, the leading cryptocurrencies, such as Bitcoin and Ethereum, function as the decentralized benchmarks, and people hold them as a way to be ``in". Both fiat and cryptocurrency benchmarks have biases toward their specific design, code, and policy decisions. For example, stablecoins depend on central bank policies while the leading cryptocurrencies depend on their respective code bases and communities. UNIT, however, averages all top cryptocurrencies to create an unbiased unit.

Because of the properties and design, UNIT sits in a privileged position to bring the cryptocurrency space together. UNIT can function as the connecting unit in the cryptocurrency space.

Venturing into the era of blockchain, cryptocurrency and web3, we often find ourselves inundated with a multitude of promising projects. Amidst this, UNIT emerges as a revolutionary solution, allowing us to track the entire crypto market, reducing the need to focus on individual projects, and eliminating the stress of managing our crypto portfolio. UNIT can function as a shortcut for those who want to stay invested in Crypto in the long run.

\section{UNIT DAO}

UNIT is governed by the selection criteria, Section \ref{sec:selectionCriteria}, and index formula algorithm, Section \ref{sec:unitData}. In turn, these algorithms can be slightly changed through voting mechanisms using the governing token of UNIT, UN.

UNIT DAO has a decentralized autonomous organization structure, and like other DAOs, it is governed by the global community of contributors and speculators. 

\subsection{Governance Processes}

The governance token for UNIT, UN, is designed initially to govern the algorithm through a minimal set of governing voting processes. 

As the road map unwraps, the community will carry out a rollout on the most extensive and most widely accepted asset-enabled transactional layers, including but not limited to Ethereum and its layer-2 networks.

UN holders govern UNIT by using UN to vote on proposals. There is a grace period in which UN holders can veto an approved proposal if it is considered risky to the system. The rules are deployed on-chain through our governor contract.

Governing processes are carried out using state-of-the-art voting and governing tools such as Tally \cite{tally2023documentation}, and governor contracts are deployed to carry out the processes. The community and UN holders will be in charge of adapting the processes to the latest standards.


\section{UNIT Selection Criteria}
\label{sec:selectionCriteria}

UNIT includes all large-cap cryptocurrencies in its algorithm, both to unite and natively quantify cryptocurrency. Similarly to the mechanisms governing the consensus rules in cryptocurrencies, the global community, particularly through the UNIT DAO, maintains the inclusion criteria and formula.

Here are the original inclusion rules established through the publication of this document and its respective implementations. 

\subsection{Criteria for Inclusion in UNIT}

Let $s$ be the total current supply of the rank 1 cryptocurrency (currently Bitcoin) and $\displaystyle{\phi =\frac{1+\sqrt{5}}{2}}$, the golden ratio. Then the following requirements hold.

\begin{enumerate}

\item Valuation: The 30-days average daily capitalization must be greater than $\displaystyle{\frac{s}{\phi^{12}}\approx\frac{s}{322}}$.
\item Volume of Trade: The currency trades widely with an established consensus on its importance given by a UNIT DAO vote.
\item Fees paid to Validators: Another fundamental metric is the expenses accrued by the cryptocurrency. A UNIT DAO vote will also consider this metric when including and excluding cryptocurrencies.
\item Issuance: The consensus rules must define the currency supply.
\item Availability: As a weak rule, $50\%$ of the supply must be available for trading.

\end{enumerate}

\subsection{The UNIT Reconstitution}

The inclusion and exclusion of coins from the UNIT follow the criteria in the previous section. The mechanism also includes a monthly vote from the UNIT DAO holders. And the goal is to ensure the continuity of the UNIT.

Changes to this process and specific parameters can be achieved through proposals submitted to the UNIT DAO, which comprises global community members.

\section{UNIT Data}
\label{sec:unitData}

\subsection{Setting the Initial UNIT Level}

UNIT is retroactively set to 1 Satoshi after the Bitcoin Genesis Block. UNIT equals exactly 1 Satoshi until the incorporation into the UNIT of the second cryptocurrency, Litecoin.

\subsection{Currency Data and a Basic Index}

The index data comes from decentralized sources such as decentralized exchanges and oracle networks. This publicly verifiable data is then publicly aggregated.

The UNIT index is a market capitalization weighted index, so the total market capitalization being accounted for is the sum of the market caps of the coins in the UNIT:

$$
s_1 p_{1,j}+s_2 p_{2,j}+ \cdots+ s_n\ p_{n,j} = \sum_{i=1}^{n} s_ip_{i,j}
$$

where $s_i$ is the supply of the $i^{th}$ coin and $p_{i,j}$ is the price of the $i^{th}$ coin in terms of the $j^{th}$ coin.

We can define UNIT at time $t$, $Ø_t$, by

  $$
Ø_t = \left(\sum_{i}\frac{p_{i,t}}{p_{i,m-1}} w_{i,m-1}\right) Ø_{m-1}
  $$
where $p_{i,m}$ is the price of coin $i$ in the Unit at month $m$ and $w_{i,m}$ is the weight of coin $i$ in the Unit at month $m$. 

The set of weights is updated monthly, choosing the monthly closing market caps.

\subsection{Typical Index Elements Not Included in The UNIT}

\subsubsection{Index Normalization}

Index normalization is a standard process to create a value that does not contain any units. It usually works as follows:

$$
\frac{s_1 p_{1,j}+ s_2\ p_{2,j}+\cdots+ s_n\ p_{n,j}}{s_j}= \frac{\displaystyle{\sum_{i=1}^{n} s_ip_{i,j}}}{s_j}
$$

However, since we are interested in creating a unit of account, we will not need to normalize or delete the currency units of the index. 


\subsubsection{Index Divisor}

The divisor aims to scale down the number to a smaller unit of value. The divisor is also used to ensure the continuity of the index with changing market conditions. We originally used the product of the total global human population $N$ and the life expectancy at birth in years $Y$ to scale down UNIT. 
$$
\frac{\displaystyle{\sum_{i=0}^{n} S_iP_{i,0}}}{Y N}
$$
But since we decided to originally anchor it to 1 Satoshi we dropped the use of a divisor.

Index divisors are often used to offset the impact of the addition or deletion of coins to an index. Similarly, a divisor may be adjusted to ensure that changes in coin availability do not impact the index. However, the way we are updating UNIT does not require a divisor used in these ways. 


\subsubsection{Volatility Factors}

Indices can also include normalization factors to dampen the effects of the extreme volatility of each cryptocurrency. Currently, there is no volatility factor included.

\subsection{Fork Coins}

UNIT has a special treatment for forks of coins that happen while a coin is in the UNIT. So far there have been 5 forks of coins in the UNIT that have been added to the UNIT. Ethereum Classic (ETC) added on 2016-08, Bitcoin Cash (BCH) added on 2017-09, Bitcoin Gold (BTG) added on 2017-12, Bitcoin Diamond added on 2018-05 and Bitcoin Satoshi's Vision (BSV) added on 2018-12. Ethereum (ETH) owners automatically owned the Ethereum Classic (ETC) fork and Bitcoin (BTC) owners owned the other four forks. Given the nature of these forks, the first time they are added to the UNIT they increase the value of UNIT at no cost to holders.

\subsection{Airdrops}

Some airdrops might also need special treatment. If an airdrop is given to the holders of any coin present in the UNIT at the time of the airdrop, and the airdropped cryptocurrency is added to the UNIT, then the airdrop would increase the value of UNIT at no cost to the holders. Therefore the percentage of the total available supply that is part of the airdrop would be calculated to reflect its impact on the UNIT.

\section{TINU}

TINU is a token pegged to UNIT that can only be issued through the overcollateralized vaults governed by the UNIT DAO holders.

\subsection{UNIT Overcollateralized Vaults: TINU Minting and Burning}

UNIT Vaults are collateralized by assets agreed to by the UNIT DAO through the UN token voting. The minimum collateralization level depends on the price ranges in UNIT of each asset used in the particular vault. Each vault only contains one individual asset and a set of parameters. Our launch will include, wBTC and wstETH as the collateral sources, but UNIT governance will soon consider other collateral sources.

To mint TINU, participants must deposit collateral into UNIT vaults. Vaults have a set of risk parameters to ensure TINU’s stability and the continuity of the peg to UNIT. The risk parameters are:

\begin{itemize}
\item UNIT Fee APY: Fee that depends on the amount of TINU issued to the vault owner, currently 0.
\item Liquidation Ratio: Level at which a vault will enter liquidation.
\item Liquidation Fee: A fee paid by the vault owner for undergoing liquidation.
\end{itemize}


\subsection{Farming Incentives}

UNIT vaults are receiving early incentives through farming rewards.

Currently, farming participants will receive farming rewards in the form of new UN from the farming pool. Farms will both open vaults and ad liquidity to swap pools. Participants receive farming incentives as long as they keep swap LP tokens locked in the farming contracts.


\subsection{Closing a Vault}

To recover the collateral, vault owners must return all the TINU. Participants control their vaults as long as the collateral doesn’t fall below the liquidation level. 


\subsection{Account Liquidations}

Suppose an account's collateralization ratio falls below the liquidation ratio. In that case, the account owner will have 72 hours to bring the collateral back over the minimum requirement by adding collateral and by burning TINU. However, after 72 hours, anyone can claim collateral locked in the contract by burning the required amount of TINU. This is all done through our auction mechanism.


\subsection{Soft Peg}

TINU follows UNIT thanks to the overcollateralization of vaults and decentralized price feeds from a decentralized oracle network.

\subsection{TINU Utilities}

TINU allows market participants to expose themselves to the entire cryptocurrency market without being exposed to the risk of owning an individual coin. 

TINU also allows participants to trade cryptocurrencies away from the US dollar by using TINU as the quote currency instead of USDT, USDC, or DAI.

TINU can be lent to participants who wish to believe in the current system for some time and would like to sell TINU for USD pegs in the hopes the cryptocurrency market will fall relative to fiat currencies. Due to its nature, TINU can be lent to market participants at high-interest rates. 


\subsection{Attacks on TINU}

In the case of an attack that could compromise the TINU system, the community will activate the Emergency Shutdown. The Emergency Shutdown will paralyze the attacker and allow TINU holders to claim all the remaining collateral.

\subsection{Full Decentralization}

The decentralization process starts with the first node, the first participant, and ends with global participation in a censorship-resistant protocol. UNIT has been following this road from the beginning.

\section{Engagement and Market Participants}

\subsection{UNIT Prediction Market and Prize Pools}

To get people acquainted with how UNIT works and the importance of size and relative importance ranking, we plan to open up pools to bet on the outcome of a ranking event.

Pools are designed to create a global cross-blockchain game around the ranking of coins and the components of UNIT.

These pools can deposit on lending platforms such as Aave to increase the rewards and automatically assign prizes to the winners.

\subsection{Third Party Index Funds}

Because this is a decentralized index, index fund providers can create their pegs to UNIT. These pegs should increase the relevance of UNIT over time. Private index funds verified once per $n$ blocks could also become a product. 


\section{Disclaimer}

UNIT is not an investment product. It is not possible to invest directly in an index. The past performance of an index is not an indication or guarantee of future results. UNIT makes no assurance that investment products based on the index will accurately track index performance or provide positive investment returns. A decision to invest in any such investment fund or other investment vehicles should not rely on the statements outlined in this document. 

These materials have been prepared solely for informational purposes based upon information generally available to the public and from sources believed to be reliable. In no event shall UNIT Parties be liable to any party for any direct, indirect, incidental, exemplary, compensatory, punitive, special, or consequential damages, costs, expenses, legal fees, or losses (including, without limitation, lost income or lost profits and opportunity costs) in connection with any use of the Content even if advised of the possibility of such damages.


\begin{thebibliography}{1}

\bibitem{nakamoto2008bitcoin}
Satoshi Nakamoto,
\newblock Bitcoin: A Peer-to-Peer Electronic Cash System,
\newblock 2008,
\newblock \url{https://bitcoin.org/bitcoin.pdf},
\newblock Accessed: 2024-06-11.

\bibitem{compound2019whitepaper}
Compound Finance,
\newblock Compound: The Money Market Protocol,
\newblock 2019,
\newblock \url{https://compound.finance/documents/Compound.Whitepaper.pdf},
\newblock Accessed: 2024-06-11.

\bibitem{aave2020whitepaper}
Aave,
\newblock Aave Protocol Whitepaper v1.0,
\newblock 2020,
\newblock \url{https://github.com/aave/aave-protocol/blob/master/docs/Aave_Protocol_Whitepaper_v1_0.pdf},
\newblock Accessed: 2024-06-11.

\bibitem{uniswap2020whitepaper}
Uniswap,
\newblock Uniswap Whitepaper,
\newblock 2020,
\newblock \url{https://uniswap.org/whitepaper.pdf},
\newblock Accessed: 2024-06-11.

\bibitem{eip721}
William Entriken, Dieter Shirley, Jacob Evans, Nastassia Sachs,
\newblock ERC-721 Non-Fungible Token Standard,
\newblock 2018,
\newblock \url{https://eips.ethereum.org/EIPS/eip-721},
\newblock Accessed: 2024-06-11.

\bibitem{uf2020methodology}
Banco Central de Chile,
\newblock Unidad de Fomento (UF) Methodology,
\newblock 2020,
\newblock \url{https://si3.bcentral.cl/estadisticas/Principal1/Metodologias/EMF/UF.pdf},
\newblock Accessed: 2024-06-11.

\bibitem{imf2023sdr}
International Monetary Fund,
\newblock Special Drawing Rights (SDR) Factsheet,
\newblock 2023,
\newblock \url{https://www.imf.org/en/About/Factsheets/Sheets/2023/special-drawing-rights-sdr},
\newblock Accessed: 2024-06-11.

\bibitem{tally2023documentation}
Tally,
\newblock Technical Documentation,
\newblock 2023,
\newblock \url{https://docs.tally.xyz/#technical-documentation},
\newblock Accessed: 2024-06-11.

\end{thebibliography}


\end{document}